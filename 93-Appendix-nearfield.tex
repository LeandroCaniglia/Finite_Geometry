\chapter{Nearfields}

\section{The Hall Nearfield}

\begin{defn}
    A \textsl{nearfield} is an algebraic structure $(N,+,\cdot)$, similar to a field, but with one key axiom weakened:
    \begin{enumerate}[a)]
        \item $(N, +)$ is an abelian group (the addtivive group of the nearfield).
        \item $(N \setminus\set0,\cdot)$ is a group (the \textsl{multiplicative} group of the nearfield).
        \item The zero element acts trivially for multiplication: $0\cdot x=0$.
        \item Right distributivity holds: $(a+b)\cdot c= a\cdot c + b\cdot c$ for all $a,b,c\in N$.
    \end{enumerate}
\end{defn}

\begin{rem}
    Note that in a nearfield left distributivity does \textit{not} necessarily hold: $a\cdot(b+c)$ is not necessarily equal to $a\cdot b + a\cdot c$. This is the defining difference from a division ring or field.
\end{rem}

\begin{xmpl}\label{The Hall Nearfield} \citep[Example 11.1.4.iv, pp.~317]{Stevenson1992}
    Let $\Fq$ denote, as usual, the finite field with $q$ elements. Fix $f(x)=x^2-rx-s$ irreducible in $\Fq{[x]}$, and let $\omega$ be a root of~$f$ in~$\Fq[q^2]$. Then
    $$
        \omega^2=s+r\omega.
    $$
    The \textsl{Hall nearfield} $H(q^2)$ is defined using the elements and addition of the field $\Fq[q^2]$, and changing the multiplication as follows: Given $a=a_0+a_1\omega$ and $b=b_0+b_1\omega$, their Hall product is defined by twisting the first coordinate of the regular product $a\cdot b$ in $\Fq[q^2]$ in the following way: the equality
        \begin{align*}
            a\cdot b &= (a_0+a_1\omega)(b_0+b_1\omega)\\
                &= a_0b_0+(a_0b_1+a_1b_0)\omega + a_1b_1(s+r\omega)\\
                &= a_0b_0\underline{{}+a_1b_1s}
                    +(a_0b_1\overline{{}+a_1b_0}+a_1b_1r)\omega,
        \intertext{becomes}
            (a_0+a_1\omega)\circ(b_0+b_1\omega)
                &= \begin{cases}
                    a_0b_0+a_1b_0\omega
                        &b_1=0,\\
                    a_0b_0\underline{{}-a_1b_1^{-1}f(b_0)}
                        +(a_0b_1\overline{{}-a_1b_0}+a_1r)\omega
                        &b_1\ne0
            \end{cases}
        \end{align*}
    \textbf{Right distribution:} is satisfied since the multiplication is clearly bilinear in the first operand.
    
    \textbf{Left distribution:} is not generally satisfied because the multiplication is not bilinear in second operand (it is not linear in any of the components of the second operand).
\end{xmpl}