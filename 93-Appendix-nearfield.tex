\chapter{Nearfields}
\label{ch:nearfields}

\section{Quasifields}

\begin{defn}\label{defn:quasifield}
    A triple $(R,{}\+{},\cdot)$ is called a \textsl{quasifield} if the following axioms hold:
    \begin{enumerate}[label=qf$_{\arabic*}$,font=\scshape]
        \item\label{QF1} $(R,\+)$ is a group with identity element $0$.
        \item\label{QF2} For every $a,b\in R\setminus\{0\}$, the equations
            \[
                a\cdot x = b
                \qquad\text{and}\qquad
                x\cdot a = b
            \]
            admit unique solutions $x\in R\setminus\{0\}$.
        \item\label{QF3} There exists an element $1\in R$ such that $1\cdot x = x\cdot 1 = x$ for all $x\in R$.
        \item\label{QF4} $0\cdot x = x\cdot 0 = 0$ for all $x\in R$.
        \item\label{QF5} For all $a,b,c,d\in R$ with $a\ne c$, there exists a unique $x\in R$ such that
            \[
                x\cdot a \+ b = x\cdot c \+ d.
            \]
        \item\label{QF6} For all $a,b,c,d\in R$ with $a\ne c$, there exists a unique pair $(x,y)\in R\times R$ such that
            \[
                a\cdot x \+ y = b,
                \qquad
                c\cdot x \+ y = d.
            \]
        \item\label{QF7} \textsl{Right distributivity}: For all $a,b,c\in R$,
            \[
                (a\+ b)\cdot c = a\cdot c \+ b\cdot c.
            \]
    \end{enumerate}
    If condition~\textsc{\ref{QF7}} is omitted, $(R,\+,{}\cdot{})$ is called a \textsl{neardomain}.
\end{defn}

\begin{xmpl}{\upshape[Moulton Neardomain]}
    Define in the reals the operation
    \[
        x\circ y=\begin{cases}
            xy  &\text{if }x\ge0\text{ or }y\ge0,\\[1mm]
            \dfrac{xy}2    &\text{if }x,y<0.
        \end{cases}
    \]
    The triple $(\R,+,\circ)$ is a neardomain and not a quasifield.

    Axiom \textsc{\ref{QF1}} is trivial.

    For axiom \textsc{\ref{QF2}}, since the multiplication is commutative, we only need to consider the equations $a\circ x=b$. Its solutions are
    \[
        \begin{array}{c|cc}
            a\circ x=b&b>0 &b<0\\
            \hline\rule{0mm}{3.5mm}
            a>0&ba^{-1}&ba^{-1}\\
            a<0&ba^{-1}&2ba^{-1}
        \end{array}
    \]

    Axioms \textsc{\ref{QF3}} and \textsc{\ref{QF4}} are trivial.

    For axiom \textsc{\ref{QF5}}, it suffices to consider the equation $x\circ a=x\circ b+c$ with $a>b$. By \textsc{\ref{QF2}} we may assume that $a,b\ne0$. Then,
    \[
        \begin{array}{c|cc}
            &c>0 &c<0\\
            \hline\rule{0mm}{3.5mm}
            b>0&c(a-b)^{-1}&c(a-b)^{-1}\\
            a<0&c(a-b)^{-1}&2c(a-b)^{-1}\\
            b<0<a&c(a-b)^{-1}&c(a-b/2)^{-1}
        \end{array}
    \]

    In axiom \textsc{\ref{QF6}}, after replacing $y$ with $y-d$, we may assume that $a>c$ and $d=0$. This leaves us with
    \[
        y = -(c\circ x)\quad\text{and}\quad
        a\circ x- c\circ x=b
    \]
    which is equivalent to \textsc{\ref{QF5}} because the multiplication is commutative.

    Finally, note that \textsc{\ref{QF7}} does not hold: right distributivity fails, for instance, when $c<0$ and $-a<b<0<a$.
\end{xmpl}

\begin{rem}\label{rem:qf7-implies-qf6}
    If \textsc{\ref{QF7}} is satisfied and $(R,\+)$ is abelian, axiom \textsc{\ref{QF6}} is a direct consequence of \textsc{\ref{QF2}} because the equation $(a\+ c^-)\cdot x=b\+ d^-$, where the superscript `${}^-$' denotes the inverse in $(R,\+)$, can be solved for $x$ and then, substitution into the first equation, would give~$y$.
\end{rem}

\begin{xmpl}\label{xmpl:hall-quasified} {\upshape[The Hall quasifield]} \citep[Example 11.1.4.iv, pp.~317]{Stevenson1992}
    Let $\Fq$ denote, as usual, the finite field with $q$ elements. Fix a polynomial $f(x)=x^2-rx-s$ irreducible in $\Fq{[x]}$, and let $\omega$ be a root of~$f$ in~$\Fq[q^2]$. Then
    $$
        \omega^2=s+r\omega.
    $$
    The \textsl{Hall quasifield} $H(q^2)$ is defined using the elements and addition of the field $\Fq[q^2]$, and changing the multiplication as follows: Given $a=a_0+a_1\omega$ and $b=b_0+b_1\omega$, their Hall product is defined by twisting the first coordinate of the regular product $a\cdot b$ in $\Fq[q^2]$ in the following way: the equality
        \begin{align*}
            a\cdot b &= (a_0+a_1\omega)(b_0+b_1\omega)\\
                &= a_0b_0+(a_0b_1+a_1b_0)\omega + a_1b_1(s+r\omega)\\
                &= a_0b_0\ulcolor{blue}{{}+a_1b_1s}
                    +(a_0b_1\ulcolor{teal}{{}+a_1b_0}+a_1b_1r)\omega,
        \intertext{becomes}
            a\circ b
                &= \begin{cases}
                    a_0b_0+a_1b_0\omega
                        &b_1=0,\\
                    a_0b_0
                        \ulcolor{blue}{{}-a_1b_1^{-1}f(b_0)}
                        +(a_0b_1\ulcolor{teal}{{}-a_1b_0}+a_1r)\omega
                        &b_1\ne0
            \end{cases}
        \end{align*}
    Axiom \textsc{\ref{QF1}} is trivially satisfied.

    For axiom~\textsc{\ref{QF2}}, consider the equation $a\circ x = b$, which can be written as
    \[
        \left\{\begin{aligned}
            x_1 &= 0,\\
            b_0 &= a_0x_0,\\
            b_1 &= a_1x_0
        \end{aligned}\right.
        \qquad\text{or}\qquad
        \left\{\begin{aligned}
            b_0 &= a_0x_0 - a_1x_1^{-1}f(x_0),\\
            b_1 &= a_0x_1 - a_1x_0 + a_1r.
        \end{aligned}\right.
    \]
    The first system admits a necessarily unique solution precisely when $a_0b_1 = a_1b_0$.
    
    Assume now that $a_0b_1 \ne a_1b_0$ and let's show that the second system has a unique solution:
    \begin{itemize}
        \item If $a_1 \ne 0$, we first solve the second equation for $x_0$ and substitute into the first, obtaining $x_1$.
        \item If $a_1 = 0$, we solve the first equation for $x_0$ and then determine $x_1$ from the second.
    \end{itemize}
    In either case, uniqueness follows immediately from the form of the equations.

    To prove that $x\circ a=b$ also admits a unique solution, we must analyze these two possibilities:
    \[
        \left\{\begin{aligned}
            a_1 &= 0,\\
            b_0 &= a_0x_0,\\
            b_1 &= a_0x_1
        \end{aligned}\right.
        \qquad\text{or}\qquad
        \left\{\begin{aligned}
            a_1 &\ne 0,\\
            b_0 &= a_0x_0 - a_1^{-1}x_1f(a_0),\\
            b_1 &= a_1x_0 - a_0x_1 + x_1r.
        \end{aligned}\right.
    \]
    The first system has a necessarily unique solution precisely when $a_1=0$. If $a_1\ne0$, we can solve the last equation of the second system for $x_0$ obtaining
    \[
        x_0=a_1^{-1}b_1+a_1^{-1}(a_0-r)x_1.
    \]
    Substituting into the first equation,
    \[
        b_0=a_0a_1^{-1}+\big(a_0a_1^{-1}(a_0-r)-a_1^{-1}f(a_0)\big)x_1,
    \]
    which can be solved for $x_1$ provided that
    \[
        \bcancel{a_0^2}-\cancel{a_0r}-\bcancel{a_0^2}+\cancel{ra_0}
            +s\ne0,
    \]
    which does hold because $x^2-rx-s$ is irreducible.

    Axioms \textsc{\ref{QF3}} and \textsc{\ref{QF4}} are clearly satisfied.

    To verify axiom \textsc{\ref{QF5}} consider the equation
    \[
        x\circ a+ b=x\circ c+d
    \]
    where $a\ne c$. The conclusion is clear if $a_1=0$ and $c_1=0$. Since the sum is associative we may assume that $d=0$. There are (essentially) two cases:
    \begin{itemize}
        \item $a_1\ne0$ and $c_1=0$. Here, we have
        \begin{align*}
            a_0x_0-a_1^{-1}x_1f(a_0)+b_0 &= c_0x_0\\
            a_1x_0-a_0x_1+x_1r+b_1 &= c_0x_1.
        \end{align*}
        Solving the second equation for $x_0$ and substituting into the first equation, we can then solve for $x_1$.
        \item $a_1\ne0$ and $c_1\ne0$. In this case we have
        \begin{align*}
            a_0x_0-a_1^{-1}x_1f(a_0)+b_0
                &= c_0x_0-c_1^{-1}x_1f(c_0)\\
            a_1x_0-a_0x_1+\cancel{x_1r}+b_1
                &= c_1x_0-c_0x_1+\cancel{x_1r}.
        \end{align*}
        If $a_0\ne c_0$, we can solve the second equation for $x_1$ and substitute it into the first, obtaining $x_0$.

        If $a_0=c_0$, the system reduces to
        \begin{align*}
            -a_1^{-1}x_1f(a_0)+b_0 &= -c_1^{-1}x_1f(a_0)\\
            a_1x_0+b_1 &= c_1x_0.
        \end{align*}
        We can solve the second for $x_0$ and then use the first to obtain~$x_1$.
    \end{itemize}
    Axiom \textsc{\ref{QF7}} is satisfied since the multiplication is clearly bilinear in the first operand. By Remark~\ref{rem:qf7-implies-qf6}, \textsc{\ref{QF6}} is also satisfied.
    
    Note that \textit{left distribution} is not generally satisfied because the multiplication defined is not bilinear in the second operand (it is not even linear in any of the components of the second operand).

    \textit{Associativity} is never satisfied. To see this it is enough to consider the case $b_1\ne0$ and $c_1=0$. Here we have $c_0=c$ and
    \begin{align*}
        (a\circ b)\circ c
            &= a_0b_0c - a_1b_1^{-1}cf(b_0)
                +(a_0b_1c-a_1b_0c+a_1cr)\omega\\
        a\circ(b\circ c)
            &= a_0b_0c-a_1b_1^{-1}c^{-1}f(b_0c)
                + (a_0b_1c-a_1b_0c+a_1r)\omega.
    \end{align*}
    Thus, associativity holds only if $r=0$ and $cf(b_0) = c^{-1}f(b_0c)$. Now,
    \begin{align*}
        cf(b_0) = c^{-1}f(b_0c) &\iff
        c^2(b_0^2-s) = b_0^2c^2-s\\
        &\iff c^2s = s\\
        &\iff c^2=1,
    \end{align*}
    which implies $q=2$. However, $x^2-s$ is not irreducible in $\Fq[2][x]$.
\end{xmpl}

\begin{xmpl}
    Consider $\Fq[3^2]$ as the quotient $\Fq[3][x]/\gen{x^2+1}$ and change the multiplication to
    \[
        a\circ b=\begin{cases}
            ab      &\text{if $b$ is a square in $\Fq[3^2]$},\\
            a^3b    &\text{otherwise}.
        \end{cases}
    \]
    Then $(N_{3^2},+,\circ)$ whose the underlying set is $\Fq[3^2]$ with the sum of $\Fq[3^2]$ and the multiplication defined above, is a right quasifield, and not a nearfield.

    Let $i$ denote a root of $x^2+1$. Then
    \[
        N_{3^2}=\set{a_0+a_1i\mid a_0,a_1\in\Fq[3]}
            =\set{0,1,2,i,1+i,2+i,2i,1+2i,2+2i}.
    \]
    The squares in $\Fq[3^2]$ are $\set{a_0+a_1i\mid a_0a_1=0}$. Indeed,
    \[
        \begin{array}{c|ccccccccc}
            a&0&1&2&i&1+i&2+i&2i&1+2i&2+2i\\
            \hline\rule{0pt}{12pt}
            a^2&0&1&1&-1&-i&i&-1&i&-i\\
            %a^3&0&1&2&2i&1+2i&2+2i&i&1+i&2+i
        \end{array}
    \]
    Since
    \[
        (a_0+a_1i)^3= a_0^3+a_1^3i^3=a_0-a_1i,
    \]
    we obtain
    \begin{align*}
        (a_0+a_1i)\circ b &= a_0b + a_1bi\\
        (a_0+a_1i)\circ bi &= -a_1b+a_0bi\\
        (a_0+a_1i)\circ(b_0+b_1i) &= (a_0-a_1i)(b_0+b_1i)
                &&;\ b_0,b_1\in\set{1,-1}\\
            &= a_0b_0+a_1b_1+(a_0b_1-a_1b_0)i.
    \end{align*}
    If we define $f(x)=x^2+1$, the last expression can be rewritten as
    \[
        a_0b_0-a_1b_1^{-1}f(b_0)+(a_0b_1-a_1b_0)i,
    \]
    because $f(b_0)=f(\pm1)=-1$ and $b_1^{-1}=b_1$. On the other hand, when $b_0=0$ and $b_1=b=\pm1$, the \rhs\ of the second equality above becomes
    \[
        -a_1b_1+a_0b_1i,
    \]
    which we can also rewrite it as
    \[
        a_0b_0-a_1b_1^{-1}f(b_0)+(a_0b_1-a_1b_0)i.
    \]
    Therefore, this quasifield is actually $H(3^2)$. As we have seen in Example~\ref{xmpl:hall-quasified}, multiplication associativity does not hold. In fact, if $c$ is not a square in $\Fq[3^2]$, we have
    \[
        (a\circ b)\circ c = a^3b^3c
        \quad\text{and}\quad
        a\circ(b\circ c)=a\circ(b^3c).
    \]
    Choose $b=c$. Then $b^3c=(c^2)^2$ and so
    \[
        (a\circ b)\circ c = a^3c^4
        \quad\text{and}\quad
        a\circ(b\circ c) = ac^4,
    \]
    which are distinct for $c\ne0$ and $a^3\ne a$.
\end{xmpl}

\section{Nearfields}

\begin{defn}
    A \textsl{nearfield} is a quasifield whose first operation is commutative and its second operation is associative.    
\end{defn}
    
\begin{rem}
    A \textsl{nearfield} $(N,+,\cdot)$ is an algebraic structure, similar to a field, but with one key axiom weakened:
    \begin{enumerate}[a)]
        \item $(N, +)$ is an abelian group (the addtivive group of the nearfield).
        \item $(N \setminus\set0,\cdot)$ is a group (the \textsl{multiplicative} group of the nearfield).
        \item The zero element acts trivially for multiplication: $0\cdot x=0$.
        \item Right distributivity holds: $(a+b)\cdot c= a\cdot c + b\cdot c$ for all $a,b,c\in N$.
    \end{enumerate}
\end{rem}

\begin{rem}
    Note that in a nearfield left distributivity does \textit{not} necessarily hold: $a\cdot(b+c)$ is not necessarily equal to $a\cdot b + a\cdot c$. This is the defining difference from a division ring or field.
\end{rem}

\begin{xmpl}
        Algebras $M$ and $A$, which follow, are examples of neardomains.
\end{xmpl}