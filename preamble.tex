\usepackage[utf8]{inputenc}
\usepackage[greek,english]{babel}
\usepackage{lmodern}
\usepackage{amsthm}
\usepackage{etoolbox}
%\usepackage{graphicx}
\usepackage{parskip}
\usepackage{amsmath}
\usepackage[shortlabels]{enumitem}
%\usepackage{supertabular}
\usepackage{mathtools}
\usepackage{amssymb}
\usepackage{array}
%\usepackage{booktabs} % Required for \cmidrule
\usepackage{hyperref}
%\usepackage{enumitem}
\usepackage{tikz}
\usepackage{tikz-cd}
\tikzset{close/.style={outer sep=-2pt}}
\usetikzlibrary{arrows.meta, positioning, calc}
\usetikzlibrary{intersections}
\usetikzlibrary{patterns}
\usetikzlibrary{matrix,arrows,decorations.pathmorphing}
\usepackage{mathrsfs} %for calligraphic text
\usetikzlibrary{angles,quotes}
\usetikzlibrary{calc}
%\usepackage{ifthen}
\usepackage{chngcntr}
%\usepackage{caption}
%\usepackage{subdepth}
%\usepackage[mathletters]{ucs}
%\usepackage[utf8x]{inputenc}
%\usepackage{newunicodechar}
\usepackage{cancel}
\usepackage{enumitem}
\usepackage{needspace}
%\usepackage{tabularray}
\usepackage[T1]{fontenc}
%\usepackage{mathrsfs}
%\usepackage[outline]{contour}
%   \contourlength{0.1pt}
%   \contournumber{10}
%\usepackage [cal=dutchcal]{mathalfa}

\usepackage[dvipsnames]{xcolor}

\usepackage[square,sort,comma]{natbib}

%\usepackage{arydshln}
\usepackage{xspace}

\hypersetup{
    colorlinks=true,
    linkcolor=darkgray,
    citecolor=black,  % Set citation color to black
    urlcolor=darkgray
    }
\usepackage{stmaryrd}
\usepackage{graphicx} % Required for inserting images

\usepackage{listings}     % The main package for code listings
% \usepackage{amssymb}      % For symbols like \mathbb{N}
% \usepackage{amsmath}      % For symbols like \mapsto, \land
\usepackage{textcomp}     % For upquote option if needed
%\usepackage{lmodern}
 
%{\fontfamily{cmss}\selectfont
%This text uses a different font typeface
%}

%NEW COMMANDS
\newcommand{\T}[1]{#1^{\rm T}}
\newcommand{\col}[1]{\mathbf{#1}}
\newcommand{\row}[1]{\T{\col#1}}
\newcommand{\dotp}[2]{\row{#1}\col{#2}}
\newcommand{\N}{\mathbb{N}}
\newcommand{\R}{\mathbb{R}}
\newcommand{\C}{\mathbb{C}}
\newcommand{\Z}{\mathbb{Z}}
\newcommand{\Q}{\mathbb{Q}}
\newcommand{\A}{\mathbb{A}}
\newcommand{\Proj}[1]{\mathbb{P}^#1}
\DeclareMathOperator{\E}{\mathbb{E}}
\newcommand{\op}[1]{\operatorname{#1}}
\newcommand{\Rn}{\R^n}
\newcommand{\Rm}{\R^m}
\newcommand{\Rk}{\R^k}
\newcommand{\Rd}{\R^d}
\newcommand{\Rnxn}{\R^{n\times n}}
\newcommand{\Rmxn}{\R^{m\times n}}
\newcommand{\Rkxn}{\R^{k\times n}}
\newcommand{\Rnxm}{\R^{n\times m}}
\newcommand{\Sph}[1]{\mathbb S^{#1}}
\newcommand{\Sn}{\Sph n}
\newcommand{\D}[1]{\mathbb D^{#1}}
\newcommand{\csum}{\mathbin{\ooalign{$\hidewidth\#\hidewidth$\cr$\phantom{+}$}}}

%\newcommand{\conj}[1]{\overline{#1}}
\newcommand{\conj}[1]{\widebar{#1}}

\newcommand{\norm}[1]{\Vert{#1}\Vert}
\newcommand{\abs}[1]{\left\vert{#1}\right\vert}
\newcommand{\noend}{\ifmmode\to\leftarrow\else$\to\leftarrow$\fi}
\DeclareMathOperator{\tr}{tr}
\newcommand{\rank}[1]{\op{rank}#1}
\newcommand{\ev}[1]{{\rm ev}_{#1}}
\newcommand{\im}[1]{\op{im}#1}
\newcommand{\id}{\op{id}}
\newcommand{\dom}{\mathrm{dom}}
\newcommand{\epi}{{\rm\bf epi}}
\newcommand{\cl}{\mathrm{cl}}
\newcommand{\hypo}{{\rm\bf hypo}}
\newcommand{\prob}{{\rm prob}}
\newcommand{\diag}{{\op{diag}}}
\newcommand{\sg}{\op{sg}}
\DeclareMathOperator{\supp}{supp}
\newcommand{\alg}{\fontfamily{cmss}\selectfont}
\newcommand{\since}{\Longleftarrow{}}
\DeclareMathOperator*{\liminfd}{lim\,inf\vphantom{\limsup}}
\newcommand{\overbar}[1]{\,\overline{\!{#1}}}
\DeclareMathOperator{\argmin}{argmin}
\DeclareMathOperator{\Cov}{Cov}
\newcommand{\seq}[2][n]{(#2)_{#1\in\N}}
\newcommand{\set}[1]{\{#1\}}
\newcommand{\blank}{\vbox{\medskip\medskip}}
\newtheorem{thm}{\blank Theorem}[section]
\newtheorem{ntn}[thm]{\blank Notation}
\newtheorem{ntns}[thm]{\blank Notations}
\newtheorem{lem}[thm]{\blank Lemma}
\newtheorem{cor}[thm]{\blank Corollary}
\newtheorem{prop}[thm]{\blank Proposition}
\newtheorem{probl}[thm]{\blank Problem}
\newtheorem{exr}[thm]{\blank Exercise}
\newtheorem{ill}[thm]{Illustration}
\theoremstyle{definition}
\newtheorem{defn}[thm]{Definition}
\newtheorem{defns}[thm]{Definitions}

\theoremstyle{definition}
    \newtheorem{rem}[thm]{\blank Remark}
    \newtheorem{rems}[thm]{\blank Remarks}
    \newenvironment{solution}[1][]
      {\begin{proof}[Solution#1]${}$}
      {\end{proof}}
    \newtheorem{xmpl}[thm]{Example}
    \newtheorem{xmpls}[thm]{Examples}
    \newtheorem{test}[thm]{\blank Test}
\theoremstyle{plain}

\AtBeginEnvironment{test}{\small}
\AtEndEnvironment{test}{\normalsize}


\newcounter{num}
\setcounter{num}{1}
\newcommand{\incnum}{\setcounter{num}{\value{num}+1}}
\newcommand{\resetnum}{\setcounter{num}{1}}
\newcommand{\num}{\arabic{num}\incnum}
\newcommand{\numlabel}[1]{\newcounter{#1}\setcounter{#1}{\value{num}}}
%\newcommand{\qed}{\hfill$\quad\square$}
\newcommand{\preferbreak}{\vfil\penalty-50\vfilneg}
\newcommand{\widechar}[2]{\DeclareUnicodeCharacter{#1}{#2}}
%\newcommand{\proof}{\vskip\parskip{\it Proof.\ }}
\counterwithin*{equation}{section}
\counterwithin*{equation}{subsection}
\newcommand\Ccancel[2][black]{%
  {%
    \renewcommand\CancelColor{\color{#1}}%
    \cancel{#2}%
  }%
}
\newcommand{\mbf}[1]{\textrm{\boldmath$#1$}}

% probability commands
\DeclareMathOperator{\borel}{{\cal B}}
\DeclareMathOperator{\diff}{\triangle}
\newcommand{\nset}[2][1]{\llbracket#1,#2\rrbracket}
\newcommand{\Lb}[1]{{\cal L}^{#1}}
\newcommand{\barLb}[1]{\bar{\cal L}^{#1}}
\DeclareMathOperator{\esup}{{\rm esssup}}
\newcommand{\aew}{,{\rm\ a.e.}}
\newcommand{\as}{,{\rm\ a.s.}}
\newcommand{\io}{,{\rm\ i.o.}}
\newcommand{\eqd}{\stackrel{({\rm d})}{=}}
\newcommand{\toin}[1]{\stackrel{\vphantom{|}#1}{\to}}
\newcommand{\toas}{\toin{{\rm a.s.}}}
\newcommand{\toinLp}[1][p]{\toin{\Lb #1}}
\newcommand{\tow}{\toin{w}}
\newcommand{\toD}{\toin{D}}
\newcommand{\toP}{\toin{\Prob}}
\DeclareMathOperator{\Prob}{\mathbb{P}}
\DeclareMathOperator{\Exp}{{\rm Exp}}
\DeclareMathOperator{\Var}{{\rm Var}}
\DeclareMathOperator{\Int}{\rm int}
\newcommand{\given}{\,|\,}
\newcommand{\CE}[2][G]{\E[#2\given{\cal #1}]}
\newcommand{\CProb}[2][G]{\Prob[#2\given{\cal #1}]}
\DeclareMathOperator{\Gram}{Gram}
\DeclareMathOperator{\vol}{vol}


% topology
\newcommand{\freeh}[1]{\lceil{#1}\rceil}
\newcommand{\fcong}{\stackrel{\cong}{F}}
\newcommand{\inner}[2]{\langle#1,#2\rangle}

% diff geom
\newcommand{\kn}{\kappa_{\col n}}
\newcommand{\kg}{\kappa_{\col g}}
\newcommand{\smooth}[1][\infty]{C^{#1}\!}
\newcommand{\cvD}[1]{\operatorname D_{#1}}
\newcommand{\cvd}[2][S]{\lfloor\operatorname D#2\rfloor_#1}
\newcommand{\scvd}[2][S]{\mbf(\!\operatorname D#2\mbf)_#1}
\newcommand{\iprod}{\!\mathbin{\lrcorner}}
\DeclareMathOperator{\vf}{\mathfrak{X}}

\makeatletter
\newcommand{\superimpose}[2]{%
  {\ooalign{$#1\@firstoftwo#2$\cr\hfil$#1\@secondoftwo#2$\hfil\cr}}}
\makeatother
\newcommand{\tcap}{\mathbin{\mathpalette\superimpose{{\cap}{|}}}}

% widebar (https://tex.stackexchange.com/a/60253/102342)
    \makeatletter
    \let\save@mathaccent\mathaccent
    \newcommand*\if@single[3]{%
      \setbox0\hbox{${\mathaccent"0362{#1}}^H$}%
      \setbox2\hbox{${\mathaccent"0362{\kern0pt#1}}^H$}%
      \ifdim\ht0=\ht2 #3\else #2\fi
      }
    %The bar will be moved to the right by a half of \macc@kerna, which is computed by amsmath:
    \newcommand*\rel@kern[1]{\kern#1\dimexpr\macc@kerna}
    %If there's a superscript following the bar, then no negative kern may follow the bar;
    %an additional {} makes sure that the superscript is high enough in this case:
    \newcommand*\widebar[1]{\@ifnextchar^{{\wide@bar{#1}{0}}}{\wide@bar{#1}{1}}}
    %Use a separate algorithm for single symbols:
    \newcommand*\wide@bar[2]{\if@single{#1}{\wide@bar@{#1}{#2}{1}}{\wide@bar@{#1}{#2}{2}}}
    \newcommand*\wide@bar@[3]{%
      \begingroup
      \def\mathaccent##1##2{%
    %Enable nesting of accents:
        \let\mathaccent\save@mathaccent
    %If there's more than a single symbol, use the first character instead (see below):
        \if#32 \let\macc@nucleus\first@char \fi
    %Determine the italic correction:
        \setbox\z@\hbox{$\macc@style{\macc@nucleus}_{}$}%
        \setbox\tw@\hbox{$\macc@style{\macc@nucleus}{}_{}$}%
        \dimen@\wd\tw@
        \advance\dimen@-\wd\z@
    %Now \dimen@ is the italic correction of the symbol.
        \divide\dimen@ 3
        \@tempdima\wd\tw@
        \advance\@tempdima-\scriptspace
    %Now \@tempdima is the width of the symbol.
        \divide\@tempdima 10
        \advance\dimen@-\@tempdima
    %Now \dimen@ = (italic correction / 3) - (Breite / 10)
        \ifdim\dimen@>\z@ \dimen@0pt\fi
    %The bar will be shortened in the case \dimen@<0 !
        \rel@kern{0.6}\kern-\dimen@
        \if#31
          \overline{\rel@kern{-0.6}\kern\dimen@\macc@nucleus\rel@kern{0.4}\kern\dimen@}%
          \advance\dimen@0.4\dimexpr\macc@kerna
    %Place the combined final kern (-\dimen@) if it is >0 or if a superscript follows:
          \let\final@kern#2%
          \ifdim\dimen@<\z@ \let\final@kern1\fi
          \if\final@kern1 \kern-\dimen@\fi
        \else
          \overline{\rel@kern{-0.6}\kern\dimen@#1}%
        \fi
      }%
      \macc@depth\@ne
      \let\math@bgroup\@empty \let\math@egroup\macc@set@skewchar
      \mathsurround\z@ \frozen@everymath{\mathgroup\macc@group\relax}%
      \macc@set@skewchar\relax
      \let\mathaccentV\macc@nested@a
    %The following initialises \macc@kerna and calls \mathaccent:
      \if#31
        \macc@nested@a\relax111{#1}%
      \else
    %If the argument consists of more than one symbol, and if the first token is
    %a letter, use that letter for the computations:
        \def\gobble@till@marker##1\endmarker{}%
        \futurelet\first@char\gobble@till@marker#1\endmarker
        \ifcat\noexpand\first@char A\else
          \def\first@char{}%
        \fi
        \macc@nested@a\relax111{\first@char}%
      \fi
      \endgroup
    }

%groups
\newcommand{\grp}[1]{\langle#1\rangle}
\DeclareMathOperator{\ord}{ord}
\newcommand{\Sym}[1]{{\rm Sym}#1}
\newcommand{\Alt}[1]{{\rm Alt}#1}
%\newcommand{\Aut}[1]{{\rm GL}#1}
\DeclareMathOperator{\Aut}{Aut}
\newcommand{\Elt}{\mathrm{Elt}}
\newcommand{\Hom}{{\rm Hom}}
\newcommand{\End}[1]{{\op{End}}#1}
\newcommand{\Inn}[1]{{\rm Inn}#1}
\newcommand{\Syl}[1]{{\rm Syl}#1}
\newcommand{\Hall}[1]{{\rm Hall}#1}
\newcommand{\Core}{{\rm Core}}
\newcommand{\normal}{\lhd}
\newcommand{\snormal}{\mathrel{\triangleleft\triangleleft}}
\newcommand{\comm}{\mathrel\leftrightarrow}
\newcommand{\cnjcls}[1]{\left[#1\right]}
\newcommand{\spec}[1]{\operatorname{spec}#1}
\DeclareMathOperator{\lcm}{lcm}
\DeclareMathOperator{\ad}{ad}
\DeclareMathOperator{\ct}{\operatorname{ct}}
\newcommand{\subgroup}{\leqslant}
\DeclareMathOperator{\CD}{{\cal CD}}
\DeclareMathOperator{\Nm}{{\cal N}_m}
\newcommand{\Soc}[1]{{\rm Soc}#1}
\newcommand{\sd}{{\rm sd}}
\newcommand{\ch}{\mathrel{\textcolor{black!65}\blacktriangleleft}}
\newcommand{\Fpn}[1][n]{\mathbb{F}_{p^{#1}}}
\newcommand{\Fq}[1][q]{\mathbb{F}_{#1}}
\newcommand{\nnormal}{\mathrel{\ooalign{\hss$/$\hss\cr$\normal$}}}
\DeclareMathOperator{\Hol}{Hol}
%\newcommand{\lcosets}[2]{#1\!:\!#2}
\newcommand{\lco}[2]{#1\mathbin:#2}

\DeclareMathSymbol{\skw}{\mathbin}{letters}{"3F}

\newcommand{\cat}[1]{\mathbf{#1}}
\newcommand{\xto}[1]{\xrightarrow{#1}}


\newcommand{\V}{\mathbb V}
\newcommand{\W}{\mathbb W}
\newcommand{\U}{\mathbb U}
\newcommand{\Si}{\mathbb S}
\newcommand{\TT}{\mathbb T}
\newcommand{\M}{\mathbb M}
%\newcommand{\NN}{\mathbb N}
\newcommand{\J}{\mathbb J}
%\newcommand{\grp}[1]{\langle#1\rangle}
\DeclareMathOperator{\Ann}{\op{Ann}}
\newcommand{\GL}[2]{\op{GL}(#1,#2)}
\newcommand{\fchar}{\op{char}}
\newcommand{\basis}[1]{\mathcal #1}
\newcommand{\lsp}[1]{\langle#1\rangle}
\newcommand{\gen}[1]{\mbf(#1\mbf)}

\newcommand{\separator}{
\vspace{0.5em} % Space above the line
\centerline{\rule{0.2\textwidth}{0.5pt}} % Centered line
\vspace{1em} % Space below the line
}


% Geometry

\newcommand{\pts}{\mathcal{P}}
\newcommand{\blocks}{\mathcal{B}}
%\newcommand{\incidence}{\textrm{\upshape\textsc{i}}}
\newcommand{\incidence}{\mbf\imath}
\newcommand{\igeo}{(\pts,\blocks,\incidence}

\let\oldpts\pts
\let\oldblocks\blocks
\let\oldinc\incidence

\renewcommand{\pts}[1][]{\oldpts#1}
\renewcommand{\blocks}[1][]{\oldblocks#1}
\renewcommand{\incidence}[1][]{\mathrel{{\oldinc#1}}}

\usepackage{centernot}

% Base definition
%\newcommand{\incidence}{\mathrel{\textsc{i}}}

% Extended version with optional argument
%\let\oldinc\incidence
\renewcommand{\incidence}[1][]{\mathrel{{\oldinc#1}}}

% Negated version
\newcommand{\nincidence}[1][]{\centernot{\incidence[#1]}}

\renewcommand{\igeo}[1][]{%
  (\pts#1,\,\blocks#1,\,\incidence#1)}

\newcommand{\block}[1][]{\mathsf#1}

\newcommand{\mols}{\op{\textsc{mols}}}


\newcommand{\iq}{\mathsf{i}}
\newcommand{\jq}{\mathsf{j}}
\newcommand{\kq}{\mathsf{k}}

\newcommand{\lhs}{\textsc{lhs}\xspace}
\newcommand{\rhs}{\textsc{rhs}\xspace}

\newcommand{\persp}[3]{\block#1\wedge_#2\block#3}

\newcommand{\vect}[1]{\mathtt#1}
\DeclareMathOperator{\FixOp}{\mathsf{Fix}}
\newcommand{\Fix}[1]{\FixOp(#1)}

\DeclareMathOperator{\PG}{PG}
\DeclareMathOperator{\AG}{AG}
\DeclareMathOperator{\pl}{pl}
\newcommand{\rr}{\hyperref[eq:rotation-rule]{rotation rule}}
\newcommand{\qr}{\hyperref[rem:quad-rule]{quad rule}}
\newcommand{\ter}[1][R]{\block #1}
\definecolor{lightblue}{RGB}{120, 220, 255}
